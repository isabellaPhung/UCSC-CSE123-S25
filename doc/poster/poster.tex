% Gemini theme
% https://github.com/anishathalye/gemini

\documentclass[final]{beamer}

% ====================
% Packages
% ====================

\usepackage[T1]{fontenc}
\usepackage{lmodern}
\usepackage[orientation=portrait,size=a0,scale=1.15]{beamerposter}
\usetheme{gemini}
\usecolortheme{gemini}
\usepackage{graphicx}
\graphicspath{{../common/personas}{../common/}{images}}
\usepackage{booktabs}
\usepackage{tikz}
\usepackage{pgfplots}
\pgfplotsset{compat=1.14}
\usepackage{anyfontsize}
\usepackage{smartdiagram}
\usesmartdiagramlibrary{additions}

% ====================
% Lengths
% ====================

% If you have N columns, choose \sepwidth and \colwidth such that
% (N+1)*\sepwidth + N*\colwidth = \paperwidth
\newlength{\sepwidth}
\newlength{\colwidth}
\setlength{\sepwidth}{0.025\paperwidth}
\setlength{\colwidth}{0.45\paperwidth}

\newcommand{\separatorcolumn}{\begin{column}{\sepwidth}\end{column}}

% pandoc provides this command for lists
\newcommand{\pandocbounded}[1]{#1}
\providecommand{\tightlist}{%
  \setlength{\itemsep}{0pt}\setlength{\parskip}{0pt}}

% ====================
% Title
% ====================

\title{Standalone device for personal organization}
\author{
  Mason~Becker
  \and
  Sulaiman~Islam
  \and
  Isabella~Phung
  \and
  Akanksha~Rajagopalan
  \and
  Lennan~Tuffy
}
\institute[UC Santa Cruz]{CSE 123 - Supervised by Prof. David Harrison}
\date{\today}

% ====================
% Footer (optional)
% ====================

\footercontent{
  \href{https://engineering.ucsc.edu}{https://engineering.ucsc.edu} \hfill
  Senior Design Showcase 2025, UC Santa Cruz \hfill
  % FIXME who's contact info will we use?
  \href{mailto:ltuffy@ucsc.edu}{ltuffy@ucsc.edu}}
% (can be left out to remove footer)

% ====================
% Logo (optional)
% ====================

% use this to include logos on the left and/or right side of the header:
% \logoright{\includegraphics[height=7cm]{logo1.pdf}}
\logoright{\includegraphics[height=7cm]{BE_Logomark_CMYK_color.pdf}}
% \logoleft{\includegraphics[height=7cm]{logo2.pdf}}

% ====================
% Body
% ====================

\begin{document}

\begin{frame}[t]
\begin{columns}[t]
\separatorcolumn

\begin{column}{\colwidth}

  \begin{block}{Need Statement}
    \input{build/need_statement_md}
  \end{block}

  \begin{block}{Goal Statement}
    \input{build/goal_statement_md}
  \end{block}


  \begin{block}{System Architecture} %formerly Data Flow

    Our device processes user input through a simple but effective data flow:

    \begin{itemize}
      \item \textbf{Input}: User uses touch screen to navigate menus, edit data, and enter focus mode
      \item \textbf{Processing}: Custom microcontroller updates task database in real-time  
      \item \textbf{Display}: 7.5" LCD and webserver show current task/event list and status
      \item \textbf{Storage}: RTC maintains devices track of time even when powered off, and the onboard SD card maintains all user data between resets.
    \end{itemize}

    The webapp allows the user to edit activities and view the calendar and actvities. While the standalone device allows the user to view and mark their goals as complete.

  \end{block}
    
   \begin{block}{Hardware design}
    \begin{center}
      \includegraphics{PrototypeDesign.jpg}
    \end{center}

    Functional Prototype Hardware breakdown 

    \begin{enumerate}
      \item \textbf{Pin dedication}: ESP32C3 pins were dedicated to select, reset, right, left, up, and down buttons. They were also dedicated to the LCD screen,
      and the real time clock. Pin management was one of the greater limitations of the prototype and one of the aspects that necessitates a custom dedicated microcontroller.
      \item \textbf{User Input}: The buttons were the primary mode of user input. Within the LCD tablet the menus that are included are Habits, Tasks and Events. A highlight would
      indicate the cursor position in the screen and the selection. Selecting an activity opened the option to enter focus mode which would open a timer and minimalist page. The options to 
      reset the timer and end focus mode would be open within this menu and the time left would be displayed. Tasks and events are displayed on the main page with up to 4 being displayed at a time.
      The user could scroll down to enter a new page with four new tasks and continue until all existing tasks were viewed. Selecting the habits page would open a menu organized by days of the week
      and task. The user could select days to represent fullfullment of the habits for those respective days. 
    \end{enumerate}


  \end{block}

  \begin{block}{Data Flow}
    \vskip 0.5cm
    \begin{center}
      \includegraphics[width = 0.8 \linewidth]{data_flow.png}
    \end{center}
  \end{block}
\end{column}

\separatorcolumn

\begin{column}{\colwidth}
  \begin{block}{Server Architecture}
    \vskip 0.5cm
    \begin{center}
      \includegraphics[width = 0.8 \linewidth]{web_server.png}
    \end{center}
  \end{block}

    \begin{block}{Prototype Design}
    \textbf{Entries}

    For the purpose of allowing the user to achieve their goals and maintain a healthy routine, the device includes three tools to tackle the various aspects of one's day:

    \begin{center}
      \includegraphics{entry_logic.png}
    \end{center}

    \textbf{Server-Client Modularization}

    The device is designed to work within the strict limitations of the ESP32C3 board while achiving both a direct seperation from the user's computer and an accurate display of the user's entires:

    \begin{enumerate}
      \item \textbf{Server Implemenentation}. For any alteration the user is able to handle their data on the coorsponding webserver. Any entries placed into this server will be accessable to the user's device.
      \item \textbf{Syncronization}. Periodically or at the user's discretion, the device achieves a full syncronization of the user's tasks an entires. First applying any alterations made from the device to the server's database, then retrieving the remaining altered information from the server with priority on entry deletion.
    \end{enumerate}

    \end{block}

    \begin{block}{User interface}

    Et rutrum ex euismod vel. Pellentesque ultricies, velit in fermentum
    vestibulum, lectus nisi pretium nibh, sit amet aliquam lectus augue vel
    velit. Suspendisse rhoncus massa porttitor augue feugiat molestie. Sed
    molestie ut orci nec malesuada. Sed ultricies feugiat est fringilla
    posuere.

    \begin{figure}
      \centering
      \begin{tikzpicture}
        \begin{axis}[
            scale only axis,
            no markers,
            domain=0:2*pi,
            samples=100,
            axis lines=center,
            axis line style={-},
            ticks=none]
          \addplot[red] {sin(deg(x))};
          \addplot[blue] {cos(deg(x))};
        \end{axis}
      \end{tikzpicture}
      \caption{Another figure caption.}
    \end{figure}

  \end{block}

  \begin{block}{Events}

    Nulla eget sem quam. Ut aliquam volutpat nisi vestibulum convallis. Nunc a
    lectus et eros facilisis hendrerit eu non urna. Interdum et malesuada fames
    ac ante \textit{ipsum primis} in faucibus. Etiam sit amet velit eget sem
    euismod tristique. Praesent enim erat, porta vel mattis sed, pharetra sed
    ipsum. Morbi commodo condimentum massa, \textit{tempus venenatis} massa
    hendrerit quis. Maecenas sed porta est. Praesent mollis interdum lectus,
    sit amet sollicitudin risus tincidunt non.

    Etiam sit amet tempus lorem, aliquet condimentum velit. Donec et nibh
    consequat, sagittis ex eget, dictum orci. Etiam quis semper ante. Ut eu
    mauris purus. Proin nec consectetur ligula. Mauris pretium molestie
    ullamcorper. Integer nisi neque, aliquet et odio non, sagittis porta justo.

    \begin{itemize}
      \item \textbf{Sed consequat} id ante vel efficitur. Praesent congue massa
        sed est scelerisque, elementum mollis augue iaculis.
        \begin{itemize}
          \item In sed est finibus, vulputate
            nunc gravida, pulvinar lorem. In maximus nunc dolor, sed auctor eros
            porttitor quis.
          \item Fusce ornare dignissim nisi. Nam sit amet risus vel lacus
            tempor tincidunt eu a arcu.
          \item Donec rhoncus vestibulum erat, quis aliquam leo
            gravida egestas.
        \end{itemize}
      \item \textbf{Sed luctus, elit sit amet} dictum maximus, diam dolor
        faucibus purus, sed lobortis justo erat id turpis.
      \item \textbf{Pellentesque facilisis dolor in leo} bibendum congue.
        Maecenas congue finibus justo, vitae eleifend urna facilisis at.
    \end{itemize}

  \end{block}


  \begin{exampleblock}{A highlighted block containing some math}

    A different kind of highlighted block.

    $$
    \int_{-\infty}^{\infty} e^{-x^2}\,dx = \sqrt{\pi}
    $$

    Interdum et malesuada fames $\{1, 4, 9, \ldots\}$ ac ante ipsum primis in
    faucibus. Cras eleifend dolor eu nulla suscipit suscipit. Sed lobortis non
    felis id vulputate.

    \heading{A heading inside a block}

    Praesent consectetur mi $x^2 + y^2$ metus, nec vestibulum justo viverra
    nec. Proin eget nulla pretium, egestas magna aliquam, mollis neque. Vivamus
    dictum $\mathbf{u}^\intercal\mathbf{v}$ sagittis odio, vel porta erat
    congue sed. Maecenas ut dolor quis arcu auctor porttitor.

  \end{exampleblock}

  \begin{block}{References}

    \nocite{*}
    \footnotesize{\bibliographystyle{plain}\bibliography{poster}}

  \end{block}

\end{column}

\separatorcolumn

\end{columns}
\end{frame}

\end{document}
